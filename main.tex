\documentclass{article}
\usepackage[german]{babel}
\usepackage{csquotes}
\usepackage[backend=biber]{biblatex}
\usepackage[a4paper,top=2cm,bottom=2cm,left=3cm,right=3cm,marginparwidth=1.75cm]{geometry}
\usepackage{fontspec}
\usepackage[colorlinks=true, allcolors=blue]{hyperref}

\setmainfont{Arial}

\addbibresource{references.bib}

\title{
    Transferarbeit
    \\St.Galler Management-Modell
    \\SBB Cargo AG}
\author{
    Jeremy Schuler\\
    Jonas Voland
}

\begin{document}
\begin{titlepage}
    \maketitle
\end{titlepage}

\setcounter{page}{2}

\begin{abstract}
    In dieser Arbeit nehmen wir die SBB Cargo AG in das Zentrum des St. Galler Managment-Modell.
    Im ersten Kapitel wird das Unternehmen kurz vorgestellt.
    Wir schauen uns die verschiedenen Umwälts
\end{abstract}

\tableofcontents

\newpage

\section{Das Unternehmen}

\subsection{Organisation}

Die SBB Cargo AG wurde 1999 gegründet und ist ein Eigenständiges Tochterunternehmen der SBB AG.
Die SBB AG besitzt 100\% der Aktienanteile und ist somit die alleinige Inhaberin der SBB Cargo AG.
Das Top Management besteht aus 8 Personen.
% Quelle: https://www.sbbcargo.com/de/unternehmen/organisation/management-organigramm.html
Der Hauptsitz ist in Olten an der Bahnhofstrasse 12.
% Quelle: https://www.sbbcargo.com/de/unternehmen/organisation/standorte.html
Die SBB Cargo AG hat wiederum auch eine Tochterfirma ChemOil Logistics AG. \cite{test}
% Quelle: https://www.sbbcargo.com/de/unternehmen/organisation/chemoil-logistics-ag.html

\subsection{Geschäftsbereiche}

Die SBB Cargo AG fokussiert sich auf den Gütertransport in der Schweiz.
Dabei bieten sie zwei Hauptbereiche and und ein Zusatzbereich an.
Einer der zwei Hauptbereiche ist das Transportangebot welches bestehen aus Wagenladungsverkehr, Ganzzüge, Kombinierter Verkehr, Bau \& Baulogistik und Entsorgung \& Recycling besteht.
% TODO: Evtl. ein Diagramm für die Übersicht 
Der zweite Hauptbereich ist das Angebot Rollmaterial, welcher aus den folgenden Teilen besteht. Flottenmanagement \& Service, Instandhaltung SBB Cargo AG und Vermietung von Rollmaterialien. 
Der Bereich Zusatzleistungen wird aufgeteilt in Bahn nahe Logistikleistungen und Services für Bahnunternehmen. 
% Quelle: https://www.sbbcargo.com/de/angebot.html

\subsection{Grösse des Unternehmens }

SBB Cargo AG deckt rund ein Siebtel des Schweizer Güterverkehrs ab.
Im Jahr 2021 hatte das Unternehmen einen Umsatz von 777 Millionen.
In Zahlen sind das 5.0 Milliarden Nettotonnenkilometer.
Die oben genannten Leistungen werden Stand 2022 von 2178 Mitarbeitern geleistet.
Die Miterbeiter zahl sinkt über die letzten 5 Jahren leicht.
% Quelle: https://reporting.sbb.ch/personal?highlighted=f8ce472b9748ab288816aa3fc328a6d2&years=1,4,5,6,7&scroll=480
Die Wirtschaftliche Lage ist brisant.
Laut dem Geschäftsbericht 2022 fährt die SBB Cargo AG einen Verlust von 187.4 Millionen CHF ein.
Im Jahr 2021 schaffte das Unternehmen noch einen Gewinn von 1.1 Millionen. 
% Quelle: https://www.sbbcargo.com/de/medien/publikationen/geschaeftsberichte.html

\subsection{Kunden/Zielgruppen}

Die Zielgruppe von SBB Cargo AG sind in erster Linie Unternehmen, die Bedarf an Gütertransport haben.
Zu den Zielgruppen gehören auch Firmen, welche Bedarf an Langstrecken Transport haben und da bei auf eine Ökologische und zuverlässige Lösung angewiesen sind.
Die Kunden von der SBB Cargo AG kommen zum grössten Teil aus der Industrie, Produktion oder Logistik.

\subsection{Konkurrenten}
Folgende 9 Unternehmen zählen zu den grössten Gütertransportunternehmen in der Schweiz nach Umsatz im Jahr 2021 und stehen somit in Konkurrenz:
\begin{itemize}
\item Lagerhäuser der Centralschweiz AG (1.84 Millionen CHF)
\item Bertschi AG (1.03 Millionen CHF)
\item Planzer Holding AG (986 Millionen CHF)
\item Galliker Transport AG (700 Millionen CHF)
\item Hupac SA (682 Millionen CHF)
\item Cargo24 AG (539 Millionen CHF)
\item Rhenus Alpina AG (476 Millionen CHF)
\item Schenker Schweiz AG (305 Millionen CHF)
\item Loomis Schweiz AG (280 Millionen CHF)
\end{itemize}
% Quelle: https://de.statista.com/statistik/daten/studie/1050561/umfrage/groesste-schweizer-guetertransportunternehmen-nach-umsatz/

\subsection{Partner / Lieferanten}
Der Strom, für den Antrieb der Züge, wird von der SBB AG grösstenteils produziert.
Ergänzend wird der Bedarf durch Beteiligungen und Bezugsverträge gesichert.
% Quelle: https://company.sbb.ch/de/sbb-als-geschaeftspartner/leistungen-evu/energie/verbrauch.html

Ein wichtiger strategischer Partner ist die Swiss Combi AG. 
An dem Unternehmen sind die Logistikdienstleister Planzer Holding AG (40\%), Camion Transport AG (40\%), Bertschi AG (10\%) und Galliker Holding AG (10\%) beteiligt.
Zusammen wollen sie den Wagenladungsverkehr und die Verlagerungspolitik aktiv vorantreiben.
% Quelle: https://news.sbb.ch/medien/artikel/123195/sbb-stellt-sich-fuer-den-gueterverkehr-der-zukunft-auf
Und PJ Messtechnik GmbH bietet System Lösungen für den Schienenverkehr an, welche die SBB Cargo AG einsetzt. 
% Quelle: https://blog.sbbcargo.com/sbb-cargo-die-rail-cargo-group-und-pj-messtechnik-arbeiten-am-intelligenten-gueterzug/

\section{Analyse des Unternehmensumfeldes}

\subsection{Umweltspähren}

\begin{itemize}
\item Umweltschutz:
Die SBB Cargo AG ist grundsätzlich was die Ökologischen Aspekte der heutigen Zeit angeht nicht im Nachtreffen wie andere Firmen.
Dies ist darauf zurückzuführen das das ganze Güternetz, welches betrieben wird, eine CO2 Ersparnis von 490 000 Tonnen pro Jahr einbringt.
% TODO: Quelle: FEHLT

\item Umweltverschmutzung:
Eine der Grössten Klima Belastungen sind die Fahrten von Diesellokomotiven auf strecken, welche nicht elektrifiziert sind.   

\item Energieverbrauch: Im Bereich des Schienenverkehrs ist die Energie auch immer ein grosses Thema da es darum geht den Strom von möglichst nachhaltigen Quellen zu beziehen, ansonsten hat man nicht die gewünschte Umweltfreundlichkeit. 
Die SBB AG sagt das sie aktuell mit 90\% Energie aus Wasserkraft fährt und 10\% Energie aus Atomstrom sind. Das Ziel der SBB AG ist die 10\% Atomstrom mit Strom aus eigenen Anlagen oder aus Anlagen, welche unter Vertrag stehen zu ersetzen.
\item Gotthard Basistunnel:
Der Umfall im Gotthard Basistunnel führte dazu, dass für eine länger Zeit die Strecke nicht befahren werden konnte.
Die Umleitung über die Panoramastrecke darf von Güterzügen nur sehr beschränkt eingesetzt werden.
Das wiederum führte dazu, dass viele Kunden zum Strassentransport gewechselt sind.
\end{itemize}


\newpage

\printbibliography


\end{document}