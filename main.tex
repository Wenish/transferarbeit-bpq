\documentclass{article}
\usepackage[german]{babel}

\usepackage{csquotes}
\usepackage[backend=biber]{biblatex}

\usepackage[a4paper,top=2cm,bottom=2cm,left=3cm,right=3cm,marginparwidth=1.75cm]{geometry}

\usepackage{fontspec}

\setmainfont{Arial}

\addbibresource{references.bib}

\title{
    Transferarbeit St.Galler Management-Modell
    \\SBB Cargo AG}
\author{
    Jeremy Schuler\\
    Jonas Voland
}

\begin{document}

\begin{titlepage}
    \maketitle
\end{titlepage}

\setcounter{page}{2}

\begin{abstract}
    In dieser Arbeit nehmen wir die SBB Cargo AG in das Zentrum des St. Galler Managment-Modell.
    Im ersten Kapitel wird das Unternehmen kurz vorgestellt.
    Wir schauen uns die verschiedenen Umwälts
\end{abstract}

\tableofcontents

\newpage

\section{Das Unternehmen}

\subsection{Organisation}

Die SBB Cargo AG wurde 1999 gegründet und ist ein Eigenständiges Tochterunternehmen der SBB AG.
Die SBB AG besitzt 100\% der Aktienanteile und ist somit die alleinige Inhaberin der SBB Cargo AG.
Das Top Management besteht aus 8 Personen.
Der Hauptsitz ist in Olten an der Bahnhofstrasse 12.
Die SBB Cargo AG hat wiederum auch eine Tochterfirma ChemOil Logistics AG. \cite{test}

\subsection{Geschäftsbereiche}

\newpage

\printbibliography


\end{document}